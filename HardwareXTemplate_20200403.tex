% GENERAL INFORMATION: HardwareX is an open access journal established to promote free and open source designing, building and customizing of scientific infrastructure (hardware). For more details on best practices for sharing open hardware see http://www.oshwa.org/sharing-best-practices/

\documentclass[11pt, letterpaper]{article}
\usepackage[utf8]{inputenc}
\usepackage[margin=1in]{geometry}
\usepackage{titlesec}
\usepackage{tabu}
\usepackage{enumitem}
\usepackage{amssymb}
\usepackage{amsmath}
\newlist{selectlist}{itemize}{2}
\setlist[selectlist]{label=$\square$,leftmargin=*,noitemsep,topsep=0pt}

\usepackage{hyperref}
\hypersetup{
    colorlinks=true,
    linkcolor=blue,
    filecolor=magenta,
    urlcolor=cyan,
}

\urlstyle{same}

% Set up the section label formatting
\titleformat{\section}[block]{\hspace{1em}\bfseries}{\thesection.}{0.5em}{}
\titleformat{\subsection}[block]{\hspace{1em}}{\thesubsection}{0.5em}{}

\begin{document}
% Create the title block
\begin{flushleft}

% Remove all text in italics when filling out the template and replace with
% your manuscripts corresponding text in regular font.
\textit{Text in italics are template instructions. Remove and replace all
	instructions with regular font text.}

\setlength{\parindent}{0pt}
\setlength{\parskip}{10pt}
% \textbf{\large HardwareX article template}

%Insert title
%Max. 20 words. A good title should contain the fewest possible words that
%adequately describe the content of a paper.
\textbf{Title:} OpenHardwareExG: an open-hardware biopotential data capture device

%Insert Authors
\textbf{Authors:} Stephanie Medlock, Eric Herman, Kendrick Matthew Shaw

%Insert Affiliations
\textbf{Affiliations:} Amsterdam University Medical Center loc. AMC, University of Amsterdam, Department of Medical Informatics, Amsterdam Public Health Research Institute, Meibergdreef 9, Amsterdam, the Netherlands.

Foundation for Public Code vereniging, Keizersgracht 617, 1017 DS, Amsterdam, the Netherlands

MariaDB Foundation, Gamla \r{A}bov\"{a}gen 38, 02700 Grankulla, Finland
%LATIN CAPITAL LETTER A WITH RING ABOVE: U+00C5
%LATIN SMALL LETTER A WITH DIAERESIS: U+00E4

Department of Anesthesia, Critical Care and Pain Medicine, Massachusetts General Hospital, Harvard Medical School, 55 Fruit Street, Boston, MA, USA.


%Insert Contact Email
%Include institutional email address of the corresponding author
\textbf{Contact email:} s.k.medlock@amsterdamumc.nl

%Insert Abstract
%Max. 200 words. Remember that the abstract is what readers see first in electronic abstracting and indexing services. This is the advertisement of your article. Make it interesting, and easy to be understood. Be accurate and specific, keep it as brief as possible.
\textbf{Abstract:} \textit{Max. 200 words. Remember that the abstract is what readers see first in electronic abstracting and indexing services. This is the advertisement of your article. Make it interesting, and easy to be understood. Be accurate and specific, keep it as brief as possible.}

%Insert Keywords
% At least 3 keywords. There is no limit on the no. of keywords you can list. Please remember that effective keywords should not repeat words appearing in your title, and should be neither too general nor too narrow.
\textbf{Keywords:} \textit{At least 3 keywords. There is no limit on the no. of keywords you can list. Please remember that effective keywords should not repeat words appearing in your title, and should be neither too general nor too narrow.}

\newpage
\textbf{Specifications table:}

\tabulinesep=1ex
\begin{tabu} to \linewidth {|X|X[3,l]|}
\hline  \textbf{Hardware name} & OpenHardwareExG rev 1 \\
  \hline \textbf{Subject area} & Medical, Electrophysiology, Biopotential measurement\\
  \hline \textbf{Hardware type} & Measuring physical properties and in-lab sensors, EEG, ECG, EMG \\
\hline \textbf{Open source license} &
 Hardware: CERN open hardware license.\cite{CERNOHL} Software: GPLv3\cite{GPLv3} \\
\hline \textbf{Cost of hardware} &
  %TODO calculate after making BOM
  %Approximate cost of hardware (complete breakdown will be included in the Bill of Materials).
  \textit{Approximate cost of hardware (complete breakdown will be included in the Bill of Materials).}
  \\
\hline \textbf{Source file repository} &
  DOI: \text{https://doi.org/10.17605/OSF.IO/42TKV}
\linebreak
\\\hline
\end{tabu}
\end{flushleft}

\section{Hardware in context}
% Include a short description of the hardware, putting into context of similar open hardware and proprietary equipment in the field.
EEG, ECG etc are important medical devices
also useful for biopotential research, e.g. BCI, sea slugs
cost of devices is a barrier, need for open hardware
whatever groovy things HardwareX has published recently

\textit{Include a short description of the hardware, putting into context of similar open hardware and proprietary equipment in the field.}
\section{Hardware description}
% Describe the hardware, highlighting the customization rather than the steps of the procedure. Highlight how it differs/which advantage it offers over pre-existing methods. For example, how could this hardware: be compared to other hardware in terms of cost or ease of use, be used in the development of further designs in a particular area, and so on.

% > Add 3-5 bulleted points to broadly explain to other researchers how the hardware could be potentially useful to them, for either standard or novel laboratory tasks, inside or outside of the original user community.

%Describe the hardware, highlighting the customization rather than the steps of the procedure. Highlight how it differs/which advantage it offers over pre-existing methods. For example, how could this hardware: be compared to other hardware in terms of cost or ease of use, be used in the development of further designs in a particular area, and so on. \linebreak \linebreak Add 3-5 bulleted points to broadly explain to other researchers how the hardware could be potentially useful to them, for either standard or novel laboratory tasks, inside or outside of the original user community.
\begin{itemize}
\item story runs from the familiar to the exotic (arduino/usb -> EEG signals)
\item SAM3X; USB 2.0 interface utility of Arduino compatibility
\item Importance of electrical isolation; measures taken
\item ADS1299 ADC
\item antialiasing filters
\item challenges of biopotential signals
\item patient disconnect lights
\item top boards/choice for touchproof connectors
\item mention firmware and arduino shield extensibility in passing
\end{itemize}

\begin{itemize}
\item Able to measure signals on the order of microvolts with input impedances of a few kiloohms
\item Reinforced electrical isolation of analog inputs for safety of patient/subject
\item Open hardware; low cost compared to commercial alternatives
\item Compatible with Arduino software and extensible via Arduino shields
\item Uses industry standard DIN 42 802 1.5 mm touch-proof plugs for patient connections
\end{itemize}

\section{Design files}
% The  complete  design  files  must  be  either  uploaded  to  an  approved  online  repository,  uploaded  at the  time  of  submission  on  the  online  Editorial  Manager  submission  interface  as  supplementary materials [CAD files, videos,. . . ], or included in the body of the manuscript [e.g.  figures].  The three approved  online  repositories  are  Mendeley  Data,  the  Open  Science  Framework,  and  Zenodo. See repository instructions: https://doi.org/10.5281/zenodo.3346799

The design files can be found at: \text{https://doi.org/10.17605/OSF.IO/42TKV}

\subsection{Design Files Summary}
% Please include a summary of all design files for your hardware by filling rows of the table below

% We should look at moving most generated files to a top-level directory
% Consider explaining a manifest of some sort in the repo itself with this
% information plus the purpose of each file

\tabulinesep=1ex
\begin{tabu} to \linewidth {|X|X|X[1.5,1]|X[1.5,1]|}
\hline
\textbf{Design filename} & \textbf{File type} & \textbf{Open source license} & \textbf{Location of the file} \\\hline
%Insert design files
\textit{Design file 1} & \textit{e.g. CAD file, figures, videos} & \textit{All designs must be submitted under an open hardware license. Enter the corresponding open source license for the file.} & \textit{Enter a link to the online location or the sentence: ``available with the article'', as appropriate}  \\\hline
\textit{Design file 2} & \dots & \dots & \dots \\\hline
% Design file 3 & File type & License & Link \\\hline

\end{tabu}

% For each design file listed in the summary above, include a short description of the file below (one or two sentences)


\section{Bill of materials}
% For a complex Bill of Materials, the complete Bill of Materials (editable spreadsheet file e.g., ODS file type or PDF file) can be uploaded in an open access online location such as the Open Science Framework repository. Include the link here. Alternatively, the Bill of Materials can be uploaded at the time of submission on the online Elsevier submission interface as supplementary material.

% > To make it easy to tell which item in the Bill of Materials corresponds to which component in your design file(s), use matching designators in both places, or otherwise explain the correspondence.

% > For material type, select from: Metal, semi-conductor, ceramic, polymer, biomaterial, organic, inorganic, composite, nanomaterial, semiconductor, non-specific, or other

%For a complex Bill of Materials, the complete Bill of Materials (editable spreadsheet file e.g., ODS file type or PDF file) can be uploaded in an open access online location such as the Open Science Framework repository. Include the link here. Alternatively, the Bill of Materials can be uploaded at the time of submission on the online Elsevier submission interface as supplementary material.

arduino-ads129x.bom \text{https://osf.io/zn8ut/}
% check VCAP capacitor values against datasheet. match datasheet or justify difference.
patch-shield-analog.bom \text{https://osf.io/p2zbe/}
patch-shield-digital.bom \text{https://osf.io/2wcev/}
% create and add BoM for laser-cut case
% consider master BoM for the three layers + case
% consider follow/transform-into suggested format of table below
% we've decided to make a BOM using the table format below. will probably use a spreadsheet to make it.

\tabulinesep=1ex
\begin{tabu} to \linewidth {|X|X|X|X|X|X|X|}
\hline
\textbf{Designator} & \textbf{Component} & \textbf{Number} & \textbf{Cost per unit currency} & \textbf{Total cost} & \textbf{Source of materials} & \textbf{Material type} \\\hline

%Building BOM in a spreadsheet. Here's the tab-separated version.
%Designator	Component	Part number	Number	Cost per unit (euro)	Total cost	Source of materials	Material type	notes
%Arduino ADS129x								
%C1-32, C99, C101	Multilayer Ceramic Capacitor MLCC - SMD/SMT 50V 270pF 0603 C0G 5%	C0603C271J5GACTU	34	0.09		Mouser	Electronic	
%C33	1500pF ±1% 50V Ceramic Capacitor C0G, NP0 0603	GRM1885C1H152FA01D	1					
%C34, C38, C44-47, C50-57, C63-66, C73-76, C80, C82-83, C86, C88-89, C91-94, C96-98, C100, C104-105, C109, C112, C119-120 	0.1µF ±5% 16V Ceramic Capacitor X7R 0603	VJ0603Y104JXJPW1BC	42					
%C35-37, C40-43, C48-49, C60-61, C67-72, C77, C81, C84, C87, C102-103, C017, C111, C115, C118, C121-123	10µF ±20% 10V Ceramic Capacitor X7R 0805	LMK212AB7106MG-T	30					
%C39	100µF Molded Tantalum Capacitors 10V 2312 200mOhm	593D107X9010C2TE3	1					
%C58-59, C62, C116	10µF ±20% 50V Ceramic Capacitor X7R 1210	UMK325AB7106MM-T	4					
%C78-79, C85, C90, C95	Ceramic multilayer 22pF 50V C0G 2% Pad SMD 0603 125°C T/R	VJ0603A220GXACW1BC	5					C106, 108, 110, 113, 114, 117 not populated?
%D1	Zener Diode 27V 1.25W ±7% Surface Mount DO-214AC	BZG03C27TR	1					
%D2-3	Schottky Diode 30V 2A Surface Mount M-FLAT	CMS06	2					
%D4, D6	LED Uni-Color Green 2-Pin Chip 0603	598-8070-107F	2					D5 not populated?
%J1	USB-B (USB TYPE-B) USB 2.0 Receptacle Connector 4 Position Through Hole, Right Angle	61729-1011BLF	1					
%L1-L6, L9-L15, L17-18, L20-22	220 Ohms @ 100MHz 1 Power, Signal Line Ferrite Bead 0805 2A 100mOhm	MH2029-221Y	18					L16, L19 not populated?
%L7-8, L23-24	47µH Shielded Wirewound Inductor 750mA 280mOhm Max 1919	SRR4028-470Y	4					
%P11a	Arduino shield header	75915-310LF 	1	1.26	1.26	Mouser	Electronic	
%P11b-c	Arduino shield header	75915-308LF	2	1.08	2.16	Mouser	Electronic	
%P11d	Arduino shield header	75915-306LF 	1	0.82	0.82	Mouser	Electronic	
%P1	header 20x2 female	929852-01-20-RB	1	3.32	3.32	Mouser	Electronic	
%P2-3	header 8x2 male	77313-101-32LF	1	1.18		Mouser	Electronic	this is a 16x2 header. I think we clipped it to make two 8x2’s?
%P4	header 5x2 male 8.08 mm posts	826656-5	1	1.17		Mouser	Electronic	
%P5	header 5x2 male	77313-101-16LF	1	0.62		Mouser	Electronic	this is an 8x2 header but our BOM says it’s 5x2
%P6	FTDI header 6x1 right angle male	5-103325-6	1	1.52		Mouser	Electronic	
%P7	JY-MCU header 4x1 right angle male	5-103325-4	1	0.81		Mouser	Electronic	
%P8	JTAG header 5x2 1.27 mm pitch male	20021121-00010C4LF	1	0.7		Mouser	Electronic	P9 and P10 not populated?
%P12	header 2x2 male	77313-101-04LF	1	0.24		Mouser	Electronic	
%P13	header 2x1 male	68000-102	1	0.21		Mouser	Electronic	
%R1-R32, R55-56	2.61 kOhms ±0.1% 0.1W, 1/10W Chip Resistor 0603 (1608 Metric) Anti-Sulfur, Automotive AEC-Q200 Thin Film	754-RG1608P-2611-BT5	34					
%R33	1 MOhms ±0.5% 0.125W, 1/8W Chip Resistor 0805 (2012 Metric) Automotive AEC-Q200 Thin Film	ERA-6AED105V	1					
%R34-37, R59	100 kOhms ±1% 0.1W, 1/10W Chip Resistor 0603 (1608 Metric) Automotive AEC-Q200, Moisture Resistant Thick Film	RK73H1JTTD1003F	5					
%R38	68.1 kOhms ±1% 0.125W, 1/8W Chip Resistor 0603 (1608 Metric) Anti-Sulfur, Automotive AEC-Q200, Moisture Resistant Thin Film	TNPW060368K1FHEA	1					
%R39, R41	39.2 kohm, ± 0.1%, 63 mW, 0603, Thin Film	PCF0603R-39K2BT1	2					
%R40	124 kOhms ±1% 0.1W, 1/10W Chip Resistor 0603 (1608 Metric) Automotive AEC-Q200 Thick Film	ERJ-3EKF1243V	1					
%R42-46, R48-51, R62-63	10 kOhms ±0.5% 0.063W, 1/16W Chip Resistor 0603 (1608 Metric) Thin Film	RR0816P-103-D	11					
%R47	1 kOhms ±1% 0.1W, 1/10W Chip Resistor 0603 (1608 Metric) Automotive AEC-Q200 Thick Film	CRCW06031K00FKEA	1					
%R52	6.8 kOhms ±1% 0.1W, 1/10W Chip Resistor 0603 (1608 Metric) Thin Film	RT0603FRE076K8L	1					
%R53-54	39 Ohms ±0.5% 0.063W, 1/16W Chip Resistor 0603 (1608 Metric) Thin Film	RR0816Q-390-D	2					R57 not populated?
%R58	560 Ohms ±0.5% 0.1W, 1/10W Chip Resistor 0603 (1608 Metric) Automotive AEC-Q200 Thin Film	ERA-3AED561V	1					
%R60-61	68 Ohms ±0.5% 0.1W, 1/10W Chip Resistor 0603 (1608 Metric) Automotive AEC-Q200 Thin Film	ERA-3AED680V	2					
%R64	8.45 kOhms ±0.5% 0.063W, 1/16W Chip Resistor 0603 (1608 Metric) Thin Film	RR0816P-8451-D-90H	1					
%R65	19.6 kOhms ±1% 0.1W, 1/10W Chip Resistor 0603 (1608 Metric) Automotive AEC-Q200 Thick Film	CRCW060319K6FKEA	1					
%R66-85	47 Ohms ±0.5% 0.1W, 1/10W Chip Resistor 0603 (1608 Metric) Automotive AEC-Q200 Thin Film	ERA-3AED470V	20					
%R86-87	20 mOhms ±1% 0.333W, 1/3W Chip Resistor 0603 (1608 Metric) Automotive AEC-Q200, Current Sense Thick Film	ERJ-3BWFR020V	2					
%SW1	Tactile Switch SPST-NO Top Actuated Surface Mount	B3SN-3012P	1					
%TH1	Polymeric PTC Resettable Fuse 60V 140mA Ih Surface Mount 1812 (4532 Metric), Concave	PTS181260V014						
%TH2	Polymeric PTC Resettable Fuse 24V 500mA Ih Surface Mount 1812 (4532 Metric), Concave	MINIASMDC050F-2						
%U1	ADS1299 8 Channel AFE 24 Bit 42mW 64-TQFP (10x10)	ADS1299IPAG 	1					
%U2-3	Linear Voltage Regulator IC 1 Output 150mA 8-MSOP-PowerPad	TPS7A4901DGNR	2					
%U4, U14	TRIAC Logic - Sensitive Gate 600V 800mA Surface Mount SC-73	BT1308W-600D	2					There is a 115 and a 135, not sure what the difference is. Is SC-73 the same as SOT223?
%U5, U18	Isolated Module DC DC Converter 1 Output 9V 222mA 4.5V - 5.5V Input	R05P209S	2					
%U6, U8	I²C Digital Isolator 5000Vrms 4 Channel 35kV/µs CMTI 16-SOIC (0.295", 7.50mm Width)	Si8606AD-B-IS						
%U7	General Purpose Digital Isolator 5000Vrms 6 Channel 150Mbps 35kV/µs CMTI 16-SOIC (0.295", 7.50mm Width)	Si8662ED-B-IS						
%U9	Buffer, Non-Inverting 1 Element 1 Bit per Element 3-State Output 5-TSSOP 	74AUP1G125GW						There are 3 different parts with this number, not sure which one we used
%U10	ARM® Cortex®-M3 SAM3X Microcontroller IC 32-Bit 84MHz 512KB (512K x 8) FLASH 144-LQFP (20x20)	ATSAM3X8EA-AU						U11 not populated?
%U12	Linear Voltage Regulator IC 1 Output 1A SOT-223	NCP1117ST50T3G						
%U13	17V Clamp 5A (8/20µs) Ipp Tvs Diode Surface Mount SOT-23-6	USBLC6-2SC6						
%U15-16	Shunt Voltage Reference IC 36V ±0.5% SOT-23-3	TL431BIDBZR						
%U17	EEPROM Memory IC 128Kb (16K x 8) I²C 1MHz 400ns 8-SOIC	CAT24C128WI-GT3						
%U19	Linear Voltage Regulator IC 1 Output 1A SOT-223	NCP1117LPST33T3G						
%X1	12MHz ±10ppm Crystal 20pF 100 Ohms 4-SMD, No Lead	ECS-120-20-33-CKM-TR						
%X2	32.768kHz ±20ppm Crystal 12.5pF 70 kOhms 2-SMD, No Lead	ABS07-32.768KHZ-T						
%Other1	shunts							I don’t remember what this is


%Insert items here
\textit{Designator 1} & \textit{Name of Component 1} & \textit{Number of units} & \textit{Cost per unit} & \textit{Total cost} & \textit{Source} & \textit{Material type} \\\hline
\textit{Designator 2} & \dots & \dots & \dots & \dots & \dots & \dots \\\hline
% Designator 3 & Name of Component 2 & Number of units & Cost per unit & Total cost & Source of materials & Material type \\\hline
\end{tabu}

\section{Build instructions}
%Provide detailed, step by step instructions for the construction of the reported hardware include all necessary information for reproducing the submitted hardware.
% > Explain and, when possible, characterize design decisions. Including design alternatives if they exist.
% > Use visual instructions such as schematics, images, and videos.
% > Clearly reference design files and component parts described in the Design File Summary and Bill of Materials.
% >Highlight potential safety concerns that may arise

\textit{Provide detailed, step by step instructions for the construction of the reported hardware
 include all necessary information for reproducing the submitted hardware.
\begin{itemize}
\item Explain and, when possible, characterize design decisions. Including design alternatives if they exist.
\item Use visual instructions such as schematics, images, and videos.
\item Clearly reference design files and component parts described in the Design File Summary and Bill of Materials.
\item Highlight potential safety concerns that may arise
\end{itemize}}

\section{Operation instructions}
%Provide detailed instructions for the safe and proper operation of the hardware.
%> Step-by-step operational instructions for operating the hardware.
%> Use visual instructions as necessary.
%> Highlight potential safety hazards.

\textit{Provide detailed instructions for the safe and proper operation of the hardware.
\begin{itemize}
\item Step-by-step operational instructions for operating the hardware.
\item Use visual instructions as necessary.
\item Highlight potential safety hazards.
\end{itemize}}

\section{Validation and characterization}
%Demonstrate the operation of the hardware and characterize its performance over relevant critical metrics
%> Demonstrate the use of the hardware for a relevant use case.
%> If possible, characterize performance of the hardware over operational parameters.
%> Create a bulleted list that describes the capabilities (and limitations) of the hardware. For example consider descriptions of load, operation time, spin speed, coefficient of variation, accuracy, precision and etc.

\textit{Demonstrate the operation of the hardware and characterize its performance over relevant critical metrics
\begin{itemize}
\item Demonstrate the use of the hardware for a relevant use case.
\item If possible, characterize performance of the hardware over operational parameters.
\item Create a bulleted list that describes the capabilities (and limitations) of the hardware. For example consider descriptions of load, operation time, spin speed, coefficient of variation, accuracy, precision and etc.
\end{itemize}}

\section{Acknowledgements}
% [List here those individuals who provided help during the research (e.g., providing language help, writing assistance or proof reading the article, etc.).] Please also identify who provided financial support for the conduct of the research and/or preparation of the article and to briefly describe the role of the sponsor(s), if any, in study design; in the collection, analysis and interpretation of data; in the writing of the report; and in the decision to submit the article for publication. If the funding source(s) had no such involvement then this should be stated.}

\textit{[List here those individuals who provided help during the research (e.g., providing language help, writing assistance or proof reading the article, etc.).] Please also identify who provided financial support for the conduct of the research and/or preparation of the article and to briefly describe the role of the sponsor(s), if any, in study design; in the collection, analysis and interpretation of data; in the writing of the report; and in the decision to submit the article for publication. If the funding source(s) had no such involvement then this should be stated.}

\section{Declaration of interest}
% a statement must be included even if there is no conflict of interest
% All authors must disclose any financial and personal relationships with other people or organizations that could inappropriately influence (bias) their work. Examples of potential conflicts of interest include employment, consultancies, stock ownership, honoraria, paid expert testimony, patent applications/registrations, and grants or other funding. Authors must disclose any interests in a summary declaration of interest statement in the manuscript file. If there are no interests to declare then please state this: 'Declarations of interest: none'. This summary statement will be ultimately published if the article is accepted. More information.}

\textit{[a statement must be included even if there is no conflict of interest] \linebreak
All authors must disclose any financial and personal relationships with other people or organizations that could inappropriately influence (bias) their work. Examples of potential conflicts of interest include employment, consultancies, stock ownership, honoraria, paid expert testimony, patent applications/registrations, and grants or other funding. Authors must disclose any interests in a summary declaration of interest statement in the manuscript file. If there are no interests to declare then please state this: 'Declarations of interest: none'. This summary statement will be ultimately published if the article is accepted. More information.}

\section{Human and animal rights}
%> If the work involves the use of human subjects, the author should ensure that the work described has been carried out in accordance with the appropriate ethical guidelines.
%> If the work involves the use of human subjects, the author should ensure that the work described has been carried out in accordance with The Code of Ethics of the World Medical Association (Declaration of Helsinki) for experiments involving humans; Uniform Requirements for manuscripts submitted to Biomedical journals. Authors should include a statement in the manuscript that informed consent was obtained for experimentation with human subjects. The privacy rights of human subjects must always be observed.
%> All animal experiments should comply with the ARRIVE guidelines and should be carried out in accordance with the U.K. Animals (Scientific Procedures) Act, 1986 and associated guidelines, EU Directive 2010/63/EU for animal experiments, or the National Institutes of Health guide for the care and use of Laboratory animals (NIH Publications No. 8023, revised 1978) and the authors should clearly indicate in the manuscript that such guidelines have been followed

\textit{
\begin{itemize}
\item If the work involves the use of human subjects, the author should ensure that the work described has been carried out in accordance with the appropriate ethical guidelines. \item If the work involves the use of human subjects, the author should ensure that the work described has been carried out in accordance with The Code of Ethics of the World Medical Association (Declaration of Helsinki) for experiments involving humans; Uniform Requirements for manuscripts submitted to Biomedical journals. Authors should include a statement in the manuscript that informed consent was obtained for experimentation with human subjects. The privacy rights of human subjects must always be observed. \item All animal experiments should comply with the ARRIVE guidelines and should be carried out in accordance with the U.K. Animals (Scientific Procedures) Act, 1986 and associated guidelines, EU Directive 2010/63/EU for animal experiments, or the National Institutes of Health guide for the care and use of Laboratory animals (NIH Publications No. 8023, revised 1978) and the authors should clearly indicate in the manuscript that such guidelines have been followed.\end{itemize}}

\section*{References}
%> Include at least one reference, to the original publication of the hardware you customized.
%> Include other references as required. Include references to put your device in context in the literature. For more information on the reference format in HardwareX please see the Guide for Authors at: https://www.elsevier.com/journals/hardwarex/2468-0672/guide-for-authors

\textit{\begin{itemize}
\item Include at least one reference, to the original publication of the hardware you customized.
\item Include other references as required. Include references to put your device in context in the literature. For more information on the reference format in HardwareX please see the Guide for Authors at: https://www.elsevier.com/journals/hardwarex/2468-0672/guide-for-authors
\end{itemize}}

\end{document}

%> Author manuscript checklist
%> ●	HardwareX is a journal dedicated to the exhaustive and fully open source communication of advances in scientific infrastructure. Upon submission the author declares that all information necessary to reproduce the subject of the submission (e.g. bill of materials, build instructions, calibration procedures, source files, code, and safety considerations) is communicated in full and is accessible for use under an open source license.
%> ●	Is the subject of the submission under an open source license - as defined by the Open Source Hardware definition?
%> ●	Can the hardware be reproduced with the details provided in the submission?
%> ●	Are all relevant design files available on Mendeley Data, the Open Science Framework, or Zenodo repositories, described in the Summary of Design Files document, and clearly documented? (e.g. descriptive file names, commented code, labeled images, etc.)
%>      ○	If in the Open Science Framework, the repository has be registered? Instructions
%>      ○	If in Zenodo, the repository is open access and is published? Instructions
%>      ○	If in Mendeley Data, the repository is published or the sharable link was included in the additional information of the Editorial Submission interface? Instructions
%> ●	Are visual instructions used when necessary?
%> ●	Is the utility of the hardware to the scientific community?
%> ●	Is the performance of the hardware adequately demonstrated and characterized?
%> ●	Are all potential safety concerns addressed?
%> ●	For more information on the article template consult the Guide to Authors.}
